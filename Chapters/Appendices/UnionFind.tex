\renewcommand\chapterformat{\thechapter\quad}
\chapter{Disjoint-Set/Union-Find Data Structure}

\paragraph{Summary}
In this chapter we briefly introduce the disjoint-set data structure and discuss
how it can be used in Kruskal's algorithm to compute minimum spanning trees.
\\

\noindent The disjoint-set data structure is a data structure that tracks a
partition of a set of elements. It provides near-constant-time operations to
\begin{itemize}
\item add new set,
\item merge existing sets, and to
\item determine whether elements are in the same set.
\end{itemize}
Since this data structure is often used in the context of graphs, the disjoint
unions of the partition are often called connected components or simply
components. We then say two elements are connected if they belong to the same
component.

Each element in the data structure stores an id, a parent pointer and possibly a
size or rank value (this is used in certain efficient algorithms). The parent
pointers are arranged to form one or more trees, each representing a set. If an
element's parent pointer points nowhere, then the element is the root of a tree
and becomes the representative member of its set. These \emph{forests} can be
represented compactly in memory as arrays in which parents are indicated by
their array index.

To make a new set from an element $x$ a new unique id is created and $x$ is
added to the disjoint-set tree. Its parent pointer is set to itself, indicating
that it is the representative member of its own set.

The find operation returns the root of the set in which an element $x$ is
contained (given it is in one of the sets). To find this root, we simply follow
the chain of parent pointers until we find a pointer that references
itself. Often, to speed up consecutive calls to find, the parent of $x$ is set
to the parent that was computed by find (this does not only speed up the find
operation for $x$ but also for elements that reference $x$).

To merge the two sets to which two elements $x$ and $y$ belong to, we first use
the find operation to obtain their respective roots (if they are the same, we
are done because then the elements already belong to the same set). Then we
attach one root to the root of the other using the parent pointers.

\subsection*{Kruskal's Algorithm}
The union-find data structure can be used to efficiently implement Kruskal's
algorithm which computes the minimum spanning tree in a graph. It is a greedy
algorithm which repeatedly finds an edge of the least possible weight that connects
any two trees in the current forest. In the beginning the forest consists of one tree for every vertex that only contains that vertex.

\begin{algorithm}
  \SetAlgoLined%
  \KwResult{Minimum spanning tree of graph $G$ with vertices $G.V$ and edges $G.E$}%
  $A \gets \emptyset$\;
  \ForEach{vertex $v$ in $G.V$}{%
    \textsc{MakeSet}(v)\;%
  }%
  \ForEach{edge $(u,v)$ in $G.E$ sorted by increasing weight}{%
    \If{\normalfont \textsc{FindSet}(u) \, $\neq$ \, \textsc{FindSet}(u)}{%
      $A \gets A \cup \{(u,v)\}$\;%
      \textsc{Union}(\textsc{FindSet}(u),\textsc{FindSet}(v))\;
    }
  }%
  \Return{$A$}
  \caption{Kruskal's algorithm}
\end{algorithm}

%%% Local Variables:
%%% mode: latex
%%% TeX-master: "../../main"
%%% End:
