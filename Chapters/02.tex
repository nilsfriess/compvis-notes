\chapter{Sensing, Subsampling and interpolation}%
\label{chap:02}
\setcounter{section}{1}
\section{On Downsampling}
To deal with very large images one often needs to downsample the
imaage so that it fits into the memory of the GPU. The \emph{wrong}
appproach would be to take the first pixels, then leave out a fixed
number, then take the next pixel and repeat.  \todo{Add Example
  images}

The effect of this na\"\i{}ve approach can be explained as follows.
Subsampling is a linear operation and can thus be written as a matrix
multiplication. If we denote by a preceding downwars arrow the signal
after subsampling, we can write
\begin{equation*}
  \downarrow g = h =
  \begin{bmatrix}
    1 & 0 & 0 & 0 & 0 & 0 & 0 & 0 \\
    0 & 0 & 1 & 0 & 0 & 0 & 0 & 0 \\
    0 & 0 & 0 & 0 & 1 & 0 & 0 & 0 \\
    0 & 0 & 0 & 0 & 0 & 0 & 1 & 0
  \end{bmatrix}
  \cdot
  \begin{bmatrix}
    g_0 \\ g_1 \\ \vdots \\ g_7
  \end{bmatrix}\,.
\end{equation*}
Every other sample is completely lost! Idea: Average two adjacent
samples. This can be written as a convolution of a blur with the
signal
\begin{equation*}
  \downarrow (b \ast g) = h^\prime = \frac{1}{2}
  \begin{bmatrix}
    1 & 1 & 0 & 0 & 0 & 0 & 0 & 0 \\
    0 & 0 & 1 & 1 & 0 & 0 & 0 & 0 \\
    0 & 0 & 0 & 0 & 1 & 1 & 0 & 0 \\
    0 & 0 & 0 & 0 & 0 & 0 & 1 & 1
  \end{bmatrix}
  \cdot
  \begin{bmatrix}
    g_0 \\ g_1 \\ \vdots \\ g_7
  \end{bmatrix}
  =
  \begin{bmatrix}
    \frac{1}{2} g_0 + \frac{1}{2} g_1 \\
    \frac{1}{2} g_2 + \frac{1}{2} g_3 \\
    \vdots \\
    \frac{1}{2} g_6 + \frac{1}{2} g_7
  \end{bmatrix}\,.
\end{equation*}
All observations contribute to the result, \ie less information is
lost. There are also other approaches. Are there optimal filters?  In
general, these filters will be data-dependent (this essentially is
PCA), \ie different filters for different data sets. To obtain a
generally appllicable filter which is not data-dependent, we need to
make some assumptions. A typical choice would be to assume the signal
to be mostly low-pass (such filters will predominantly preserve low
frequencies). A perfect low-pass filter
\begin{itemize}
\item eliminates all frequencies above the new Nyquist limit (highest
  frequency that can be represented by sampling a signal uniformly),
  and
\item leaves frequencies below the Nyquist limit completely untouched.
\end{itemize}
The optimal choice is (no proof given)
\begin{equation*}
  \text{sinc}\,x = \frac{\sin \pi x}{\pi x}
\end{equation*}
\todo{Finish this lecture}

\section{Upsampling/ Interpolating an Image}
\todo{Finish this lecture}

%%% Local Variables:
%%% mode: latex
%%% TeX-master: "../main"
%%% End:
