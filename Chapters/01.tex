\chapter{Introduction}

Computer Vision
\begin{itemize}
\item useful in consumer devices (\eg fingerprint login on smartphone)
\item is used as machine vision in quality control (\eg check that all
  pixels of a screen work $\rightarrow$ rather easy task, that there
  are no scratches $\rightarrow$ rather hard task)
\item makes life-and-death decisions (\eg in autonomous emergency
  braking, cyclist and pedestrian detection $\rightarrow$ needs
  extremely high accuracy)
  \begin{itemize}
  \item Example procedure: Take last $n$ frames (\eg $n = 11$) and
    decide based on them wether or not to brake
  \item Need extremely low false positive rate (do not want to brake
    if it is not necessary)
  \item Need low false negative rate
  \item Need to process the data in real-time, \ie need to process
    $\sim$ 30 MB input/s
  \end{itemize}
\end{itemize}
Only two hand ful of algorithms are sufficiently efficient, \ie work
at high scale and will make it into consumer devices. We will
therefore study these two hand ful of algorithms in-depth. They can
then be combined in complicated pipelines and we will study some of
these pipelines (see Table~\ref{tab:ex:tasks}).

\begin{table}[htpb]
  \centering
  \begin{tabular}{lll}
    \toprule
    Input & Output & Task \\
    \midrule
    Image & 0/1 & Image classification \\
    Image & One class per pixel & Semantic/pixel segmentation \\
    Image & Which pixel belong to which instance & Instance segmentation \\
    Image & Pose of one or more humans & Pose estimation \\
    Video & Tracks of all targets & tracking
  \end{tabular}
  \caption{Example tasks in computer vision}%
  \label{tab:ex:tasks}
\end{table}


\section{Linear Filters: Convolution}
Convolution is useful for
\begin{itemize}
\item Smoothing (Not SOTA)
\item Edge/Blob detection
\item General: Feature extraction
\end{itemize}
Has been mainstay of image analysis since it's very cheap (still
matters now) and is well-understood.

\subsection*{1D Convolution}
Consider the mean square estimator for $\{y_i\} \in \R$
\begin{gather*}
  \hat{y} = \argmin_y \sum_{i=1}^{n}{(y_i - y)}^2
\end{gather*}
\todo{Add the derviations for the different estimator from notes}

Different filters can produce very different results ($\leadsto$
filter optimazation). This is one instance of \emph{discrete
  convolution}
\begin{gather*}
  \sum_{i=0}^{n-1}f_{l-1}g_i \eqqcolon (f \ast g)_l
\end{gather*}
Properties:
\begin{itemize}
\item Convolution is \textbf{commutative}: $f \ast g = g ast f$
\item Convolution is \textbf{associative}:
  $f \ast g \ast h = f \ast (g \ast h) = (f \ast g) \ast h$
  \begin{itemize}
  \item Important in practice, especially for image analysis
  \end{itemize}
\item Convolution is distributive
\end{itemize}
Important convolution filters include
\begin{itemize}
\item $f_i \ge 0$ smoothing
\item $_i = \delta_{i-2}$ shifting
\item $f = 1/2(1 \ 0 \ -1)$ central finite difference
  (differentiation)
\end{itemize}
Note: 1D convolution can be written as matrix multiplication with a
\emph{Töplitz matrix}.  \todo{Write up example}

\subsection*{2D Convolution}
In early days only small filters were possible (\eg $f$ was 3x3).

Some filters are \textbf{seperable}: $f_{i,j} = a_i \cdot b_j$ and
this allows for storage reduction (instead of storing the full matrix
it suffices to store the vectors $a$ and $b$).

Aside: Some filters are not seperable but low rank and using singular
value decomposition we can find a suitable represantation.

In practice you would use libraries because with the right memory
layout, clever use of co-processors and GPUs speed can be increased.

There are different options to extrapolate at the boundary of the
image.
%%% Local Variables:
%%% mode: latex
%%% TeX-master: "../main"
%%% End:
